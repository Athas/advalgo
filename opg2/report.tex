\documentclass{article}
\usepackage[utf8]{inputenc}
\usepackage[danish]{babel}
\usepackage{amsmath}
\usepackage{ulem}
\usepackage{palatino}
\linespread{1.05}
\title{Advanced Algorithms - Assignment 2}
\author{Ronni Elken Lindsgaard\\
Troels Henriksen\\
Jacob Wejendorp}
\begin{document}
\maketitle
\section{Traveling Couple Problem}
For $d=0$ is TCP identical to TSP for all input graphs. And for $d > 0$
will there always be graphs where the minimum vertice distance is
greater than $d$. In which case the problem is reducable to TSP, which
is known NP-complete, therefore we cannot say wether TCP is solvable in
polynomial time.
\section{Tweak-method}
\begin{itemize}
\item The optimal datastructure is an ordered set of vertices.
Implemented in a programming language it could be either a linked list
or an indexed array.
\item $V$ is intialised to the empty set. We then start by choosing an arbitrary vertice $v$ and add it to the set $V$. 
Afterwards we take another arbitrary vertice not in the set $V$ and not
within the distance $d$ from a node in $V$. Repeat this until no such nodes are left.
\item We take two arbitrary edges $(u_1,v_1)$ and $(u_2,v_2)$ and
replace them with $(u_1,u_2)$ and $(v_1,v_2)$. The switch of directions
is propagated through the subgraph that connects $v_1$ and $u_2$.

We take a vertice $v$ from the set $V$ and look at all vertices within
range $d$ of $v$. We then choose an arbitrary vertice $u$ from this set
and check if it is possible to replace $u$ with $v$ and still have all
nodes covered.

\end{itemize}
\end{document}

