\documentclass{article}
\usepackage[utf8]{inputenc}
\usepackage[danish]{babel}
\usepackage{amsmath}
\usepackage{ulem}
\usepackage{palatino}
\usepackage{graphicx}
\usepackage{verbatim}
\linespread{1.05}
\title{Advanced Algorithms - Assignment 2}
\author{Ronni Elken Lindsgaard\and
Troels Henriksen\and
Jacob Wejendorp}
\begin{document}
\maketitle
\section{Traveling Couple Problem}
For $d=0$ TCP is identical to TSP for all input graphs. And for $d > 0$
there will always be graphs where the minimum vertice distance is
greater than $d$, in which case the problem is again equivalent to TSP, which
is known NP-complete. Therefore we cannot say whether TCP is solvable in
polynomial time.

\subsection{Data structure}

A candidate solution is a path through the graph, which we represent
as an ordered sequence of vertices.  Implemented in a programming
language it could be either a linked list or an indexed array.  We
assume that the path need not return to its origin.

\subsection{Initial candidate solution}

$V$ is intialised to the empty set. We then start by choosing an
arbitrary vertice $v$ and add it to the set $V$.  Afterwards we take
another arbitrary vertice not in the set $V$ and not within the
distance $d$ from a node in $V$.  Repeat this until no such nodes are
left.

\subsection{Mutation operators}

We make use of the following two mutation operators:

\begin{itemize}
\item We take two arbitrary edges $(u_1,v_1)$ and $(u_2,v_2)$ and
  replace them with $(u_1,u_2)$ and $(v_1,v_2)$. The switch of
  directions is propagated through the subgraph that connects $v_1$
  and $u_2$, in the form of reversing the list.
\item We take a vertice $v$ from the set $V$ and look at all vertices
  within range $d$ of $v$. We then choose an arbitrary vertice $u$
  from this set and check if it is possible to replace $u$ with $v$
  and still have all nodes covered.
\end{itemize}

\subsection{Implementation of hill-climbing}

For one hundred runs, the average cost (path length) is
$26.4827292786505$, and the minimum is $24.4202464069$.

\section{Tabu-search}
We have implemented a Tabu-search, that uses the altered nodes in the tweak procedure,
as its tabu-elements, making tabu any tweak that includes changing such a vertice.
This requires all tweaks to keep a list of nodes it alters and return it to the algorithm.
The algorithm then keeps this list, being subject to the length contraint $l$, and discards all tweaks that return
a node currently in the tabu-list.

The results using Tabu-search are slightly better than Hill-climbing,
which is primarily caused by tabus more explorative nature, which
means it will look at a larger sample, and is more likely to find any
solution.  In particular, the tabu search can pass through local
minima where the hill-climbing algorithm would get stuck.  This,
however is not unlike a simple steepest-ascent algorithm, in which the
algorithm also considers a larger neighborhood than hillclimbing. This
shows as the result-difference between a tabu-search and
steepest-ascent in our case shows little difference in results.

\subsection{Best solution}

Using $n=1000,l=10,d=0.2$, one hundred runs of the tabu search led to
an average tour length of $23.50254812367763$, a maximum of
$24.27082059618084$ and a minimum of $22.92649947513128$.

\appendix

\section{Hill-Climbing Implementation}

\verbatiminput{src/florence.py}

\section{Tabu Search Implementation}

\verbatiminput{src/taboo.hs}

\end{document}
